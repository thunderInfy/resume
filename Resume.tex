\documentclass[letter,10pt]{book}
%-----------------------------------------------------------
\usepackage[top=0.25in, bottom=0.25in, left=0.55in, right=0.55in]{geometry}
\usepackage{graphicx}
\usepackage{url}
\usepackage{palatino}
\usepackage{pdfrender}
\usepackage{tabularx}
\usepackage[colorlinks=true,linkcolor=black,anchorcolor=black,citecolor=black,filecolor=black,menucolor=black,runcolor=black,urlcolor=black]{hyperref}
\usepackage{wrapfig}
\fontfamily{SansSerif}
\selectfont

\usepackage[T1]{fontenc}
\usepackage
%[ansinew]
[utf8]
{inputenc}

\usepackage{color}
\definecolor{mygrey}{gray}{0.75}
\textheight=9.75in
\raggedbottom

\setlength{\tabcolsep}{0in}
\newcommand{\isep}{-2 pt}
\newcommand{\lsep}{-0.5cm}
\newcommand{\psep}{-0.6cm}
\renewcommand{\labelitemii}{$\circ$}

\pagestyle{empty}
%-----------------------------------------------------------
%Custom commands
\newcommand{\resitem}[1]{\item #1 \vspace{-2pt}}
\newcommand{\resheading}[1]{{\small \colorbox{mygrey}{\begin{minipage}{0.975\textwidth}{\textbf{#1 \vphantom{p\^{E}}}}\end{minipage}}}}
\newcommand{\ressubheading}[3]{
\begin{tabular*}{6.62in}{l @{\extracolsep{\fill}} r}
	\textsc{{\textbf{#1}}} & \textsc{\textit{[#2]}} \\
\end{tabular*}\vspace{-8pt}}
%-----------------------------------------------------------
\begin{document}
\begin{center}
\textsc{\Huge Aditya Rastogi}
\end{center}

\begin{minipage}{0.30\textwidth}
{\raggedleft
E-111, Azad Hall of Residence\\
Indian Institute of Technology\\
Kharagpur, West Bengal\\
India - 721302\\
}
\end{minipage}%
\begin{minipage}{0.33\textwidth}
\begin{center}
\end{center}
\end{minipage}%
\begin{minipage}{0.33\textwidth}
{\raggedright
Email-id : \textbf{r.aditya0824@iitkgp.ac.in} \\
Mobile No.: \textbf{+917477725222} \\
\href{https://github.com/thunderInfy/}{https://github.com/thunderInfy/} \\
\href{http://cse.iitkgp.ac.in/~arastogi}{http://cse.iitkgp.ac.in/$\sim$arastogi}\\
}
\end{minipage}%
\hspace{0.5cm}\\
\\
\resheading{\textbf{ACADEMIC DETAILS} }\\[\lsep]
\\
\begin{center}
\renewcommand{\arraystretch}{1.5}
\indent \begin{tabular}{ |@{\hskip 0.125in}l @{\hskip 0.125in} |@{\hskip 0.125in}l @{\hskip 0.125in} |@{\hskip 0.125in}l @{\hskip 0.125in} |@{\hskip 0.125in}l @{\hskip 0.25in} |l }
\hline
\textbf{Education} & \textbf{Institute / Board} & \textbf{Year} & \textbf{CGPA / \%} \\
\hline
B. Tech and M. Tech (Dual Degree): &&&\\[-0.5em]
{Computer Science and Engineering} & IIT Kharagpur  & 2016 - 2021 (Expected) & \textbf{9.49 / 10} \\
\hline
Class - XII & CBSE & 2016 & \textbf{96.2 \%}\\
\hline
Class - X & CBSE & 2014 & \textbf{10 / 10}\\
\hline
\end{tabular}
\end{center}
\hfill 
\hspace{0.5cm}\\
\resheading{\textbf{RESEARCH EXPERIENCE} }\\[\lsep]
\\\\
\large \textbf{Facial Landmarks Detection} \normalsize \\
\emph{Visiting Student Researcher, University of Sydney \hfill Dec'18-Jan'19 } \\ 
\emph{(Advisor: Dr. Mehar Khatkar)} \\[-2em]
\begin{itemize}
\item Compared \textbf{convolutional neural network} architectures with target point coordinates and techniques like \textbf{Constrained Local} and \textbf{Regression-based} methods in the detection of facial landmarks in different fish species.\\[-2em]
\item Used \textbf{data augmentation}, dropout and early stopping techniques to prevent overfitting on the training set data. Used R packages like \textbf{tidyverse} and \textbf{magick} to visualize the dataset effectively.\\[-1.8em]
\end{itemize}
\emph{Research Areas:}  \textbf{Deep Learning}, \textbf{Landmark Detection},
\textbf{Data Augmentation}
\\\\[-0.5em]
\large \textbf{Sensor Diagnostics} \normalsize \\
\emph{Summer Intern, Shell India Private Ltd. and IIT-Kharagpur \hfill May'18-July'18} \\ 
\emph{(Advisor: Prof. Swanand Khare)} \\[-2em]
\begin{itemize}
\item Reduced dimensionality of multiple sensors' time-series data using \textbf{PCA} and  \textbf{autoencoders}.\\[-2em]
\item Worked on the pre-image problem in  \textbf{kernel-PCA} and change-point detection methods.\\[-2em]
\item Studied Bayesian statistics and Monte-carlo-markov-chain sampling methods.\\[-2em]
\item Worked on gaussian-mixture models, \textbf{EM-algorithm} and error distributions in general.\\[-1.8em]
\end{itemize}
\emph{Research Areas:}  \textbf{Machine Learning}, \textbf{Dimensionality Reduction},
\textbf{Pre-Image Problem},  \textbf{Predictive}  \textbf{Analysis}
\\
\hspace{0cm}\\[-0.5cm]\\
\resheading{\textbf{ KEY PROJECTS } }\\[\lsep]
\\\\ \large \textbf{Sampling from Arbitrary Finite-ranged Probability Density in one dimension} \normalsize\\
\emph{(Advisor: Prof. Swanand Khare) \hfill Jun'18} \\[-2em] 
\begin{itemize}
\item Developed an interactive python drawing app using open-source Kivy library.\\[-2em]
\item Interpreted the output image as a finite-ranged probability density in one dimension using Image Processing.\\[-2em]
\item Sampling was done from the density function using  \textbf{rejection sampling methods}.\\[-1.8em]
\end{itemize}
\href{https://github.com/thunderInfy/Statistical}{https://github.com/thunderInfy/Statistical}
\\\\ \large \textbf{Interactive Algorithmic Visualisations and Web-game Development} \normalsize\\
\emph{p5-js, HTML, CSS\hfill May'18-Sept'18} \\[-2em]
\begin{itemize}
\item Developed algorithmic visualisations on the web for algorithms in \textbf{Graph Theory}, Voronoi diagram, Convex Hull, Maximum 2D Range Sum, LIS, fractals etc. using p5-js.\\[-2em]
\item Worked with \textbf{Physics Engines} like Box2D, Matter.js etc. and \textbf{genetic algorithms}.\\[-2em]
\item Developed 2D-rendered games using  \textbf{p5-js} as well.\\[-1.8em]
\end{itemize}
\href{http://cse.iitkgp.ac.in/~arastogi}{http://cse.iitkgp.ac.in/$\sim$arastogi}
\\\\
\large \textbf{Artificial Intelligence game development} \normalsize\\
\emph{Chain Reaction: AI Game Development \hfill Nov'18-Present}
\enlargethispage{4\baselineskip}\\[-2em]
\begin{itemize}
\item Developing an unbeatable single player web version of the popular android game Chain Reaction.\\[-2em] 
\item Used javascript and  \textbf{WEBGL} to render interactive 3D graphics in the web browser. \\[-2em]
\item Using \textbf{Reinforcement Learning} techniques to build the AI version.\\[-1.8em] 
\end{itemize}
\href{https://github.com/thunderInfy/Chain-Reaction-AI}{https://github.com/thunderInfy/Chain-Reaction-AI}
\pagebreak
\\
\large \textbf{Spam classifier} \normalsize\\
\emph{(Advisor: Prof. Saptarshi Ghosh) \hfill Mar'18-Apr'18} \\[-2em] 
\begin{itemize}
\item Built a deep neural network in python from scratch to classify messages as spam or not-spam.\\[-2em]
\item Applied  \textbf{porter-stemming}, used sigmoid and hyperbolic tangent as the activation functions.\\[-2em] 
\item Used  \textbf{softmax} function for probabilistic outputs.\\[-1.8em]
\end{itemize}
\href{https://github.com/thunderInfy/Spam-Classifier}{https://github.com/thunderInfy/Spam-Classifier}
\\\\
\large \textbf{Microsoft Code.Fun.Do} \normalsize\\
\emph{Servizio: Android App Development \hfill Mar'18} \\[-2em]
\begin{itemize}
\item Developed an android app which integrated the customer base with the service-providers.\\[-2em]
\item The app would provide short term employment to people and would help people in availing small services.\\[-2em]
\item Used Google firebase database cloud services for providing notifications on the app.\\[-1.8em]
\end{itemize}
\href{https://github.com/thunderInfy/Code-Fun-Do-Servizio}{https://github.com/thunderInfy/Code-Fun-Do-Servizio}
\\\\
\large \textbf{Personal Library System Software} \normalsize\\
\emph{(Advisor: Prof. Sudip Misra) \hfill Jan'18-Apr'18} \\[-2em]
\begin{itemize}
\item Developed a Java based software on personal library system. \\[-2em]
\item Used software engineering techniques and tools like \textbf{SRS} documents, \textbf{UML} class diagrams, use-case diagrams, sequence diagrams, state-chart diagrams, Java Swing, JUnit testing etc.\\[-1.8em]
\end{itemize}
\href{https://github.com/thunderInfy/Personal-Library-System-software}{https://github.com/thunderInfy/Personal-Library-System-software}
\\\\
\large \textbf{Compiler for tiny-C} \normalsize\\
\emph{(Advisor: Prof. Pralay Mitra) \hfill July'18-Nov'18} \\[-2em]
\begin{itemize} 
\item Developed a compiler for tiny-C, built front-end using  \textbf{flex} for lexer and  \textbf{yacc} for parser and back-end using C++ and \textbf{x86-64} Assembly language.\\[-1.8em]
\end{itemize}
\href{https://github.com/thunderInfy/compiler-for-tiny-C}{https://github.com/thunderInfy/compiler-for-tiny-C}
\\

\hspace{0.5cm}\\[-0.2cm]
\resheading{\textbf{TECHNICAL SKILLS} }\\[\lsep]
\\[-1em]
\begin{itemize}
\item \textbf{Languages :} Python, R, C, C++, JavaScript, Java, Verilog, MySQL, HTML, CSS, {\LaTeX{}}, MIPS Assembly
\\[-2em]
\item \textbf{Libraries and Tools :} Keras, tensorFlow, Numpy, Scipy, Matplotlib, OpenCV, Processing, p5js, tidyverse, Magick
\end{itemize}

\hspace{0.5cm}\\[-0.2cm]
\resheading{\textbf{RELEVANT COURSES} }\\[\lsep]
\\[-1em]
\begin{itemize}
\item \textbf{Completed : }*Programming and Data Structures, *Algorithms - I, Discrete Structures, Machine Learning, Formal Language and Automata Theory, *Switching Circuits and Logic Design, *Software Engineering, Probability and Statistics, *Computer Organization and Architecture, *Compilers, Algorithms - II, Data Analytics, Cryptography and Network Security 
\\[-2em]
\item \textbf{Ongoing :}  *Operating Systems, *Computer Networks, Computational Complexity, Combinatorics and Computing\\[-2em]
\item \textbf{Online courses :} \\[-2em]
\addtolength{\itemindent}{0.5cm}
\item  \textbf{Neural Networks and Deep Learning} by deeplearning.ai on coursera.\\
{*} marked courses have a laboratory component as well.
\end{itemize}

\hspace{0.5cm}\\[-0.2cm]
\resheading{\textbf{POSITIONS OF RESPONSIBILITY} }\\[\lsep]
\\[-1em]
\begin{itemize}
\item \large \textbf{Student Mentor} \hfill July'18-Present \normalsize\ \\ 
Mentored 1$^{st} $-year students under the Student Mentor Programme conducted by Student Welfare Group.
\end{itemize}

\hspace{0.5cm}\\[-0.2cm]
\resheading{\textbf{AWARDS AND ACHIEVEMENTS} }\\[\lsep]
\\[-1em]
\begin{itemize}
\item Institute topper after the first semester at IIT Kharagpur out of a total of 1324 students. 
\\[-2em]
\item Cleared regionals and participated in the nationals of Indian National Astronomy Olympiad conducted by HBCSE.
\\[-2em]
\item Secured state rank 2 in International Olympiad of English Language’ 15
conducted by SilverZone.
\\[-2em]
\item Zone Topper in Indian Intelligence Test 2015 conducted by Jagran Group.
\\[-2em]
\item Recipient of the Goralal Syngal Memorial Scholarship.
\\[-2em]
\item Voluntarily participated and actively contributed in the data collection process associated with "Assessment of Program Comprehension Skills using Eye Gaze Tracker".
\\[-2em]
\end{itemize}

\end{document}
